% Options for packages loaded elsewhere
\PassOptionsToPackage{unicode}{hyperref}
\PassOptionsToPackage{hyphens}{url}
%
\documentclass[
]{article}
\usepackage{amsmath,amssymb}
\usepackage{lmodern}
\usepackage{ifxetex,ifluatex}
\ifnum 0\ifxetex 1\fi\ifluatex 1\fi=0 % if pdftex
  \usepackage[T1]{fontenc}
  \usepackage[utf8]{inputenc}
  \usepackage{textcomp} % provide euro and other symbols
\else % if luatex or xetex
  \usepackage{unicode-math}
  \defaultfontfeatures{Scale=MatchLowercase}
  \defaultfontfeatures[\rmfamily]{Ligatures=TeX,Scale=1}
\fi
% Use upquote if available, for straight quotes in verbatim environments
\IfFileExists{upquote.sty}{\usepackage{upquote}}{}
\IfFileExists{microtype.sty}{% use microtype if available
  \usepackage[]{microtype}
  \UseMicrotypeSet[protrusion]{basicmath} % disable protrusion for tt fonts
}{}
\makeatletter
\@ifundefined{KOMAClassName}{% if non-KOMA class
  \IfFileExists{parskip.sty}{%
    \usepackage{parskip}
  }{% else
    \setlength{\parindent}{0pt}
    \setlength{\parskip}{6pt plus 2pt minus 1pt}}
}{% if KOMA class
  \KOMAoptions{parskip=half}}
\makeatother
\usepackage{xcolor}
\IfFileExists{xurl.sty}{\usepackage{xurl}}{} % add URL line breaks if available
\IfFileExists{bookmark.sty}{\usepackage{bookmark}}{\usepackage{hyperref}}
\hypersetup{
  pdftitle={Interactive GeoSpatial Visualization using R/D3},
  pdfauthor={Ravina Gelda},
  hidelinks,
  pdfcreator={LaTeX via pandoc}}
\urlstyle{same} % disable monospaced font for URLs
\usepackage[margin=1in]{geometry}
\usepackage{graphicx}
\makeatletter
\def\maxwidth{\ifdim\Gin@nat@width>\linewidth\linewidth\else\Gin@nat@width\fi}
\def\maxheight{\ifdim\Gin@nat@height>\textheight\textheight\else\Gin@nat@height\fi}
\makeatother
% Scale images if necessary, so that they will not overflow the page
% margins by default, and it is still possible to overwrite the defaults
% using explicit options in \includegraphics[width, height, ...]{}
\setkeys{Gin}{width=\maxwidth,height=\maxheight,keepaspectratio}
% Set default figure placement to htbp
\makeatletter
\def\fps@figure{htbp}
\makeatother
\setlength{\emergencystretch}{3em} % prevent overfull lines
\providecommand{\tightlist}{%
  \setlength{\itemsep}{0pt}\setlength{\parskip}{0pt}}
\setcounter{secnumdepth}{-\maxdimen} % remove section numbering
\ifluatex
  \usepackage{selnolig}  % disable illegal ligatures
\fi

\title{Interactive GeoSpatial Visualization using R/D3}
\author{Ravina Gelda}
\date{January 20, 2021}

\begin{document}
\maketitle

\textbf{Projects Overview}

\emph{Subtitle: India Population Density and Criminal Justice Data at
various Resolutions}

The Indian census provides incredibe wealth of data but its not always
easy to work with it. Working with the tabular and spatial census data
can be handled in R but getting the data, reading it into R and in
particular mering tabular and spatial data canbe a chore. Working with
slots and different classes of data and functions can be challenging.

\emph{Project A: Telangana Criminal Justice Data (2011) in R} This
project visualizes crime against women in some regions of Telangana,
India. The visualization is at a high resolution for areas covered by a
few police stations inside Telangana.

Telangana is a state in south central India created in June 2014. This
state is divided into 33 districts and 584 mandals.

\begin{verbatim}
## OGR data source with driver: ESRI Shapefile 
## Source: "D:\ravina_ucsc_project\TSDM", layer: "Mandal_Boundary"
## with 589 features
## It has 20 fields
\end{verbatim}

\begin{verbatim}
## OGR data source with driver: ESRI Shapefile 
## Source: "D:\ravina_ucsc_project\TSDM", layer: "District_Boundary"
## with 33 features
## It has 17 fields
\end{verbatim}

\hypertarget{htmlwidget-f74aad7c099f49d6eb2c}{}

\begin{verbatim}
## OGR data source with driver: ESRI Shapefile 
## Source: "D:\ravina_ucsc_project\TSDM", layer: "Mandal_Boundary"
## with 589 features
## It has 20 fields
\end{verbatim}

\hypertarget{htmlwidget-98a77937f18e957a0a01}{}

The crime data belongs to 3 police stations within the following
districts. Describe the sample crime data in words emphasizing the
variables that you will visualize (for example, lat-long etc.). The time
period of data, sample size etc.

\begin{verbatim}
##   S.No     Zone Division Police Station Crime Number Latitude Longitude
## 1    1 Bhongiri Bhongiri    Bhongir (R)    0038/2019 17.53920  78.85042
## 2    2 Bhongiri Bhongiri    Bhongir (R)    0042/2019 17.51170  79.01672
## 3    3 Bhongiri Bhongiri    Bhongir (R)    0043/2019 17.56275  78.89477
## 4    4 Bhongiri Bhongiri    Bhongir (R)    0063/2019 17.53511  78.86300
## 5    5 Bhongiri Bhongiri    Bhongir (R)    0097/2019 17.50737  79.03002
## 6    6 Bhongiri Bhongiri    Bhongir (R)    0121/2019 17.38430  78.90866
##   Present stage
## 1            PT
## 2            PT
## 3            PT
## 4   Compromised
## 5            PT
## 6   Compromised
\end{verbatim}

\textbf{Police Stations}

\begin{verbatim}
##  [1] "Bhongir (R)"     "Bhongir TN"      "Bibinagar"       "Bommala Ramaram"
##  [5] "Addaguduru"      "Atmakur (M)"     "Choutuppal"      "Mothkur"        
##  [9] "Narayanapur"     "Pochampalli"     "Ramannapet"      "Valigonda"      
## [13] "Alair"           "Motakondur"      "Rajapet"         "Thurkapalli"    
## [17] "Yadagirigutta"   "Adibatla"        "Ibrahimpatnam"   "Kandukur"       
## [21] "Madgul"          "Maheswaram"      "Manchal"         "Yacharam"       
## [25] "Chaitanyapuri"   "L.B. Nagar"      "Saroornagar"     "Abdullapurmet"  
## [29] "Balapur"         "Hayathnagar"     "Meerpet"         "Pahadisharief"  
## [33] "Vanasthalipuram" "Jawaharnagar"    "Keesara"         "Kusaiguda"      
## [37] "Neredmet"        "Ghatkesar"       "Malkajgiri"      "Medipally"      
## [41] "Nacharam"        "Uppal"
\end{verbatim}

\textbf{Zones}

\begin{verbatim}
## [1] "Bhongiri"   "LB Nagar"   "Malkajgiri"
\end{verbatim}

\textbf{DIVISIONS}

\begin{verbatim}
## [1] "Bhongiri"       "Choutuppal"     "Yadadri"        "Ibrahimpatnam" 
## [5] "Ibrahimptnam"   "LB Nagar"       "Vanastalipuram" "Kushaiguda"    
## [9] "Malkajgiri"
\end{verbatim}

\begin{verbatim}
## Warning: package 'sf' was built under R version 4.0.3
\end{verbatim}

\hypertarget{htmlwidget-dc1f28fc9f8c16ae1a84}{}

\hypertarget{htmlwidget-6a89b700e866d754fd64}{}

\hypertarget{htmlwidget-675141be54ced15706bd}{}

\textbf{Major Libraries}

\begin{itemize}
\item
  rgdal: The rgdal package provides bindings to the Geospatial Data
  Abstraction Library(GDAL) for reading, writing and converting between
  spatial formats. There are several speciality packages for reading or
  writing various formats (e.g, geojsonio,plotKML) but one function
  readOGR, one package rgdal suffices for reading spatial files. And
  hence, we used rgdal in this project to read the shape file.
\item
  ggplot2:
\item
  dplyr:dplyr is a grammar of data manipulation, providing a consistent
  set of verbs that help you solve the most common data manipulation
  challenges: mutate() adds new variables that are functions of existing
  variables select() picks variables based on their names. filter()
  picks cases based on their values. summarise() reduces multiple values
  down to a single summary. arrange() changes the ordering of the rows.
\item
  sp,sf : R has well-supported classes for sorting spatial data (sp) and
  interfacing to the above mentioned environments (rgdal, rgeos), but
  has so far lacked a complete implementation of simple features, making
  conversions at times convoluted. The package sf tries to fill the gaps
  and aims at succeeding spin long term.
\item
  leaflet: The leaflet package requires the data input to be a spatial
  object. In this project example, I started with a
  SpatialPolygonsDataFrame (which we created by reading shapefiles using
  readORG) and then I converted this to a vanilla data frame using the
  fortify function. Then can make use of your existing spatial data
  frame to plot data with leaflet.
\end{itemize}

\textbf{Challenges Faced}

\begin{itemize}
\tightlist
\item
  Main challenge was to get the shape file of telangana
\item
  Data Cleaning
\end{itemize}

\textbf{Project B: India Population Density District-Wise (2011) in R}

\textbf{Challenges}

\end{document}
